\documentclass{article}
\usepackage[utf8]{inputenc}
\usepackage{amsmath}
\usepackage{algorithmic}
\usepackage{graphicx}
\usepackage{xspace}
\usepackage{listings}



\begin{document}
\title{Hello \LaTeX\xspace World}
\author{ernieargel\\
\small{Computer Engineering and Computer Science.}\\
\small{California State University of Long Beach}\\
\small{\texttt{ernie.argel@student.csulb.edu}}
}
\date{August 31 2020}
\maketitle

\begin{abstract}
    This document is a model and instructions for \LaTeX\xspace `article' class.
\end{abstract}

\section{Introduction}
Welcome to the \LaTeX\xspacem world.

\section{Ease of Use}

\subsection{Maintaining the Integrity of the Specifications}

The `article' class is used to format your paper and style and the text. All margins, spaces, and text fonts are prescribed.

\section{Styling Guide}

\subsection{Abbreviations and Acronyms}
Define abbreviations and acronyms the first time they are used in the text. All margins, spaces, and text fonts are prescribed.

\subsection{Equations}
\begin{equation}
    \sum_{n=0}^{\infty}\frac{af^{n}}{n!}(x-a)^n
\label{taylor}
\end{equation}
\eqref{taylor} is the famous Taylor series. Use ``\eqref{taylor}'', not ``Eq.~\eqref{taylor}''or
``equation \eqref{taylor}'', except at the beginning of a sentence: ``Equation \eqref{taylor} is..."

Taylor series in a text would be $\sum_{n=0}^{\infty}\frac{af^{n}}{n!}(x-a)^n$

\subsection{Lists}
Bullet style list.
\begin{enumerate}
    \item item 1
    \item item 2
    \item item 3
\end{enumerate}

\subsection{Figures and Tables}
\paragraph{Positioning Figures and Tables} Figure captions should be below the figures; table heads should appear above the tables. Insert figures and tables after they are cited in the text. Use the abbreviation
``Fig.~\ref{fig}''.

\begin{table}[htbp]
\caption{Table Type Styles}
\begin{center}
\begin{tabular}{|c||c|c|c|}
\hline
\textbf{Table}&\multicolumn{3}{|c|}{\textbf{Table Column Head}} \\
\cline{2-4}
\textbf{Head} & \textbf{\textit{Table column subhead}}& \textbf{\textit{Subhead}}&
\textbf{\textit{Subhead}} \\
\hline
& & & \\
\hline
\end{tabular}
\label{tab1}
\end{center}
\end{table}

\begin{figure}[htbp]
\centering
\includegraphics[width=0.4\columnwidth]{fig1}
\caption{Working example}
\label{fig}
\end{figure}

\subsection{Algorithms}
\begin{algorithmic}
    \STATE $i\gets 10$
    \IF {$i\geq 5$}
    \STATE {$i\gets i-1$}
    \ELSE 
    \IF {$i\leq 3$}
    \STATE $i\gets i + 2$
    \ENDIF
    \ENDIF
\end{algorithmic}

\lstset
{
    frame=tb,
    language=Java,
    aboveskip=3mm,
    belowskip=3mm,
    showstringspace=false,
    columns=flexible,
    basicstyle={\small\ttfamily},
    numbers=none,
    breaklines=true,
    breakatwhitespace=true,
    tabsize=3
}

\subsection{Source codes}
\lstinputlisting{HelloWorld.java}

\begin{verbatim}
public class HelloWorld {
    public static void main(String[] args) {
        System.out.println("Hello, World");
    }
}
\end{verbatim}

\subsection{References}
Please number citations consecutively within brackets [1]. The sentence punctuation follows the bracket [2]. Refer simply to the reference number, as in [3]—do not use “Ref. [3]” or “reference [3]” except at the beginning of a sentence.

\begin{thebibliography}{00}
\bibitem{Eason} G. Eason, B. Noble, and I. N. Sneddon, ``On certain integrals of Lipschitz-Hankel type involving products of Bessel functions,'' Phil. Trans. Roy. Soc. London, vol. A247, pp. 529--551, April 1955.
\bibitem{Maxwell} J. Clerk Maxwell, A Treatise on Electricity and Magnetism, 3rd ed., vol. 2. Oxford: Clarendon, 1892, pp.68--73.
\bibitem{Jacobs} I. S. Jacobs and C. P. Bean, ``Fine particles, thin films and exchange anisotropy,'' in Magnetism, vol. III, G. T. Rado and H. Suhl, Eds. New York: Academic, 1963, pp. 271--350
\end{thebibliography}


\end{document}

















